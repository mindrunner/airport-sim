\documentclass[a4paper, 12pt]{scrartcl} % Font size (can be 10pt, 11pt or 12pt) and paper size (remove a4paper for US letter paper)
\input{include/header}
\usepackage{lmodern}
\usepackage{hyperref}
\usepackage{longtable}
\usepackage{abbrevs}
\usepackage[toc]{glossaries} % nomain, if you define glossaries in a file, and you use \include{INP-00-glossary}
\makeglossaries
\linespread{1.05} % Change line spacing here, Palatino benefits from a slight increase by default
\makeatletter
\renewcommand{\maketitle}{ % Customize the title - do not edit title and author name here, see the TITLE block below
  \begin{flushright} % Right align
    {\LARGE\@title}\\ % Increase the font size of the title
    \vspace{50pt} % Some vertical space between the title and author name
    {\large\@author} % Author name
    \\\@date % Date
    \vspace{40pt} % Some vertical space between the author block and abstract
  \end{flushright}
}
\title{\textbf{INX 365 Assignment 1}\\ % Title
Airport Simulator} % Subtitle
\author{\textsc{Lukas Elsner} % Author
\\{\textit{Queensland University of Technology}}} % Institution
\date{\today} % Date
\begin{document}
\maketitle % Print the title section
\thispagestyle{empty}
%\renewcommand{\abstractname}{Summary} % Uncomment to change the name of the abstract to something else
\vspace{10em}
\begin{abstract}
The task was to develop an airport simulator using C and POSIX threads. The
airport should have one runway, which is the critical resource, that cannot be
used at the same time by multiple threads. The planes should land and take-off
randomly based on given probabilities and being parked in the bay of the
airport, which can hold op to ten planes. This document covers the
implementation report, as well as occurred problems while implementing the
simulator.
\end{abstract}
\newpage
\tableofcontents
\newpage
%\hspace*{3,6mm}\textit{Keywords:} lorem , ipsum , dolor , sit amet , lectus % Keywords
%\vspace{30pt} % Some vertical space between the abstract and first section
%----------------------------------------------------------------------------------------
%	ESSAY BODY
%----------------------------------------------------------------------------------------

\section{Statement of Completeness}

All functionality was implemented as requested. Some minor additional changes
were made. This report will cover the implementation details, as well as the
additional changes.

\subsection{Basic Functionality}

The following basic functionality was implemented:
\begin{itemize}
    \item Display the banner
    \item Generate planes with given probability
    \item Randomly take-off planes with given probability
    \item Landing simulation
    \item Take-off simulation
    \item Display airport state
    \item Display "airport is empty" on empty airport
    \item Display "airport is full" on full airport
    \item Joining threads to gracefully shut down.
  \end{itemize}

In addition to that, the following extras were implemented:
\begin{itemize}
  \item Running the simulator with one or none probability argument will result
    in default values of $50$ for the non-given probabilities.
  \item Running the simulator with $-h$ parameter will print the usage help.
\end{itemize}

\subsection{Process Synchronisation and Coordination}

To coordinate and synchronize multiple threads, two semaphores and one mutes
were used. The landing and take-off threads are acting like a producer/consumer
pattern for the parking \hyperref[sec:bay]{bays} in the
\hyperref[sec:airport]{airport}. Therefore two Semaphores are used to make
sure, the threads cannot enter the critical section when their data structure
might not allow a proper action, such as landing a \hyperref[sec:plane]{plane}
onto a full \hyperref[sec:airport]{airport} and parking a
\hyperref[sec:plane]{plane} into a full \hyperref[sec:bay]{bay}.

\subsection{Non-Functionality}

Concurrency was implemented with help of the open source pthread library.
Three additional threads are spawned from the main thread:
\begin{itemize}
  \item Landing thread
  \item Take-off thread
  \item Monitor thread
\end{itemize}

To obtain an adequate readability and make it easy to maintain the source code,
many object-orientated paradigms were used for this program. Therefore, a lot
of dynamic allocation of memory was used. Listing~\ref{ValgrindAnalysis} shows
that a stress test with valgrind resulted in no possible memory leaks, even
after a long runtime of the simulator.\\
Also, the code is well commented and a doxygen reference manual was created. \\
There are no known issues regarding to unexpected behavior, such as crashes.

\lstset{language=,caption={Valgrind analysis},label=ValgrindAnalysis}
\begin{lstlisting}
==3165== HEAP SUMMARY:
==3165==     in use at exit: 0 bytes in 0 blocks
==3165==   total heap usage: 1,674,733 allocs, 1,674,733 frees, 17,307,187 bytes allocated
==3165== 
==3165== All heap blocks were freed -- no leaks are possible
==3165== 
==3165== For counts of detected and suppressed errors, rerun with: -v
==3165== ERROR SUMMARY: 1 errors from 1 contexts (suppressed: 0 from 0)
\end{lstlisting}

\section{Data structures}

\subsection{Description of Data Structures}

The airport-sim application contains three main data structures:

\subsubsection{airport}

\label{sec:airport}
Listing~\ref{AirportStruct} shows the \hyperref[sec:airport]{airport} data
structure, which is the main data structure in airport-sim. Every
\hyperref[sec:airport]{airport} has the following members.
\begin{itemize}
  \item{name} \\
    The name of the \hyperref[sec:airport]{airport}
  \item{bays} \\
    The \hyperref[sec:bay]{bays}, in which the \hyperref[sec:plane]{planes} can
    be parked.
  \item{runway} \\
    A virtual runway, to prevent multiple take-off/landing operations at the
    same time.
  \item{empty} \\
    Semaphore to block take-off thread on empty \hyperref[sec:bay]{bay}.
  \item{full} \\
    Semaphore to block landing thread on full \hyperref[sec:bay]{bay}.
\end{itemize}
\lstset{language=C,caption={Airport structure},label=AirportStruct}
\begin{lstlisting}
struct airport {
    char *name; /**< Name of the airport. */
    bay **bays;/**< Bays in which planes can be parked. Has length NUM_BAYS. */
    pthread_mutex_t runway;/**< A virtual runway, to prevent multiple
                            take-off/landing operations at the same time. */
    sem_t empty;/**< Semaphore to block on empty bay. */
    sem_t full;/**< Semaphore to block on full bay. */
};
\end{lstlisting}
Listing~\ref{AirportMethods} shows the public available methods for the
\hyperref[sec:airport]{airport} structure. The method \texttt{airport\_init()}
creates a new \hyperref[sec:airport]{airport} structure and
\texttt{airport\_destroy(airport *)} frees its memory. The methods
\texttt{airport\_land\_plane(airport *)} and
\texttt{airport\_takeoff\_plane(airport *)} are used for initiating a landing
or take-off procedure. Finally, \texttt{airport\_to\_string(airport *)}
generates the human readable string which represents the current state of the
\hyperref[sec:airport]{airport}.
\lstset{language=C,caption={Airport methods},label=AirportMethods}
\begin{lstlisting}
airport *airport_init(char *);
void airport_land_plane(airport *);
void airport_takeoff_plane(airport *);
char *airport_to_string(airport *);
void airport_destroy(airport *);
\end{lstlisting}

\subsubsection{bay}

\label{sec:bay}
Listing~\ref{BayStruct} shows the \hyperref[sec:bay]{bay} data structure, which
is used to represent a slot for a landed \hyperref[sec:plane]{plane}. Every
\hyperref[sec:bay]{bay} has the following members.
\begin{itemize}
  \item {plane} \\
    The \hyperref[sec:plane]{plane}, which is parked in this landing
    \hyperref[sec:bay]{bay}, or NULL if there is none.
  \item{parking\_time} \\
    Time, the \hyperref[sec:plane]{plane} was parked or unparked.
\end{itemize}
\lstset{language=C,caption={Bay structure},label=BayStruct}
\begin{lstlisting}
struct bay {
    plane *plane; /**< Plane parked in bay, null if there is none. */
    time_t parking_time; /**< Time, the plane was parked or unparked. */
};
\end{lstlisting}
Listing~\ref{BayMethods} shows the public available methods for the
\hyperref[sec:bay]{bay} structure. The method \texttt{bay\_init()} creates a
new \hyperref[sec:bay]{bay} structure and \texttt{bay\_destroy(bay *)} frees
its memory. The methods \texttt{bay\_park\_plane(bay *, plane *)} and
\texttt{bay\_unpark\_plane(bay *)} are used to park and unpark
\hyperref[sec:plane]{planes}. Finally, \texttt{bay\_get\_plane(bay *)} is being
used to get the currently parked \hyperref[sec:plane]{plane} of the
\hyperref[sec:bay]{bay}.
\lstset{language=C,caption={Bay methods},label=BayMethods}
\begin{lstlisting}
bay *bay_init();
time_t bay_get_occupation_time(bay *b);
void bay_park_plane(bay *b, plane *p);
plane *bay_unpark_plane(bay *b);
plane *bay_get_plane(bay *b);
void bay_destroy(bay *);
\end{lstlisting}

\subsubsection{plane}

\label{sec:plane}
Listing~\ref{PlaneStruct} shows the \hyperref[sec:plane]{plane} data structure.
Every \hyperref[sec:plane]{plane} has the following members.
\begin{itemize}
  \item{name} \\
    The name of the \hyperref[sec:plane]{plane}.
\end{itemize}
\lstset{language=C,caption={Plane structure},label=PlaneStruct}
\begin{lstlisting}
struct plane {
  char *name; /**< Name of the plane. */
};
\end{lstlisting}
Listing~\ref{PlaneMethods} shows the public available methods for the \hyperref[sec:plane]{plane}
structure. The method \texttt{plane\_init()} creates a new \hyperref[sec:plane]{plane} structure and
\texttt{plane\_destroy(plane *)} frees its memory. To get the name of a \hyperref[sec:plane]{plane},
\texttt{plane\_get\_name(plane *)} can be used.
\lstset{language=C,caption={Plane methods},label=PlaneMethods}
\begin{lstlisting}
plane *plane_init();
char *plane_get_name(plane *);
void plane_destroy(plane *);
\end{lstlisting}

\subsubsection{Time handling} In some parts of the application, it is necessary
to block the current thread for a particular amount of milliseconds. Therefore,
the method \texttt{nanosleep(struct timespec *, struct timespec *)} was used.
The \texttt{tools.c} file includes a method \texttt{msleep(long long)}, which
can be used for suspending the current thread for a particular amount of milli
seconds. The \texttt{msleep(long long)} method creates a \texttt{timespec}
structure with the required seconds ($m/1000$) and nanoseconds
($1000000*(m\%1000)$) value. This structure is passed to the
\texttt{nanosleep(struct timespec *, struct timespec *)}.

\subsection{Interaction between Threads and Data Structures}

To make sure, no race conditions are occurring between the three threads in
airport-sim, some thread synchronization patterns were used. The main critical
sections are in the methods \texttt{airport\_land\_plane(airport *)} and
\texttt{airport\_takeoff\_plane(airport *)}. Both are write accessing the
\hyperref[sec:bay]{bays} data structure. Therefore, a mutex was used to mutual
exclude the access to it.  Since this \hyperref[sec:airport]{airport} is used
with a producer/consumer pattern, a semaphore are used to let the landing
thread block when the \hyperref[sec:airport]{airport} is full and another
semaphore lets the take-off thread block when the
\hyperref[sec:airport]{airport} is empty. \\

One critical problem which happened in the special situation when the
application is going to be shut down while the \hyperref[sec:airport]{airport}
is full or empty was, that one of the threads was waiting forever for the
semaphore to change its state. Therefore the application hung on
\texttt{sem\_wait(sem\_t *)} and could not perform a clean exit. To resolve
this issue, \texttt{sem\_timedwait(sem\_t *, timespec *)} was used with a
timeout value of 5 seconds. Thus, the thread will do another loop when it
cannot acquire the semaphore after five seconds to check if the application
should exit. \\

Another race condition was found while stress-testing the application. There
was a very unlikely situation, where the take-off thread freed the memory of a
plan while the human readable output of the \hyperref[sec:airport]{airport} was
created in \texttt{airport\_to\_string(airport *)}. This could lead to a
segmentation fault and lets the application crash. To resolve this problem, the
string generation is done in a critical section now. One side effect of this
fix is, that the user might see a delay after requesting the current
\hyperref[sec:airport]{airport} state and seeing the result.

\section{Task 2}

One possible solution to support multiple runways (which obviously implies a
2-runway solution) within one \hyperref[sec:airport]{airport} would be to
introduce one additional semaphore which is representing the multiple runways.
The value of the semaphore therefore would be the amount of currently used
runways. \texttt{sem\_wait(sem\_t *)} would be called right before the
\texttt{msleep(2000)} call, which simulates landing or take-off.
\texttt{sem\_post(sem\_t *)} directly afterwards. The currently used mutex
should be renamed and used only for mutual excluding access to the parking
\hyperref[sec:bay]{bay} structure. \\

The folder \texttt{src-multi} includes a prototype for this idea. This is just
a proof of concept which might have some flaws. No additional comments were
made in the source code. \\

The following changes have been implemented:

\begin{enum}
\item Renamed mutex \texttt{runway} to \texttt{baylock}, since this is a more
  appropriate name.
\item Introduced Semaphore \texttt{runways} to simulate multiple runways.
\item Introduced mutex \texttt{runwaylock} to mutual exclude access to runway
  semaphore. This is necessary to obtain the current value of the semaphore.
  (This is only needed if we want to know which runway currently is used)
\item The methods \texttt{airport\_takeoff\_plane(airport *)} and
  \texttt{airport\_land\_plane(airport *)} were modified to more complex
  locking and synchronizing.
\item A new member for holding the runway a \hyperref[sec:plane]{plane} landed
  on was introduced in the \texttt{\hyperref[sec:bay]{bay}} data structure.
\item The method \texttt{airport\_to\_string(airport *)} was adapted to show
  the runway on which the parked \hyperref[sec:plane]{plane} was landed.
\item The main function is now able to create multiple take-off and landing
  threads based on NUM\_TAKEOFF\_THREADS and NUM\_LANDING\_THREADS.
\end{enum}

With this approach, it is possible to scale the number of threads, as well as
the size of the \hyperref[sec:airport]{airport} regarding to parking slots and
runways.

%----------------------------------------------------------------------------------------
%	BIBLIOGRAPHY
%----------------------------------------------------------------------------------------
%\bibliographystyle{IEEEtran}
%\bibliography{bibtex}
%----------------------------------------------------------------------------------------
%\printindex
%\printglossaries
%\listoffigures
%\listoftables
\end{document}
